%%% Если работа выполняется в Overleaf, 
%%% важно поменять тип компилятора
%%% с "pdfLatex" или "Latex" на "XeLaTeX". 
%%% Если этого не сделать, то и и компиляция
%%% не будет работать, и при загрузке текста ВКР
%%% на сайт МФТИ будут проблемы. 

%%% Если диплом получается очень маленький, 
%%% смело увеличивай шрифт до 14го
\documentclass[xetex,a4paper,12pt]{article} % 12й шрифт

% Подключение преамбулы с необходимыми пакетами
%%% Преамбула %%%

\usepackage{fontspec} % XeTeX
\usepackage{xunicode} % Unicode для XeTeX
\usepackage{xltxtra}  % Верхние и нижние индексы
\usepackage{pdfpages} % Вставка PDF

\usepackage{listings} % Оформление исходного кода
\lstset{
    basicstyle=\small\ttfamily, % Размер и тип шрифта
    breaklines=true,            % Перенос строк
    tabsize=2,                  % Размер табуляции
    frame=single,               % Рамка
    literate={--}{{-{}-}}2,     % Корректно отображать двойной дефис
    literate={---}{{-{}-{}-}}3  % Корректно отображать тройной дефис
}

% Шрифты, xelatex
\defaultfontfeatures{Ligatures=TeX}
\setmainfont{Times New Roman} % Нормоконтроллеры хотят именно его
\newfontfamily\cyrillicfont{Times New Roman}
\setmonofont{FreeMono} % Моноширинный шрифт для оформления кода

% Формулы
\usepackage{mathtools} % Не совместим с amsmath
\usepackage[warnings-off={mathtools-colon,mathtools-overbracket}]{unicode-math}
\setmathfont{XITS Math}             % Шрифт для формул: https://github.com/khaledhosny/xits-math
\numberwithin{equation}{section}    % Формула вида секция.номер

% Русский язык
\usepackage{polyglossia}
\setdefaultlanguage{russian}

% Абзацы и списки
\usepackage{enumerate}   % Тонкая настройка списков
\usepackage{indentfirst} % Красная строка после заголовка
\usepackage{float}       % Расширенное управление плавающими объектами
\usepackage{multirow}    % Сложные таблицы

% Пути к каталогам с изображениями
\usepackage{graphicx} % Вставка картинок и дополнений
\graphicspath{{img/}}

% Формат подрисуночных записей
\usepackage{chngcntr}

% Сбрасываем счетчик таблиц и рисунков в каждой новой главе
\counterwithin{figure}{section}
\counterwithin{table}{section}

% Гиперссылки
\usepackage{hyperref}
\hypersetup{
    colorlinks, urlcolor={black}, % Все ссылки черного цвета, кликабельные
    linkcolor={black}, citecolor={black}, filecolor={black},
    pdfauthor={Иван Чудинов},
    pdftitle={Разработка модели артериального тромбоза на основе гидрогеля}
}

% Оформление библиографии и подрисуночных записей через точку
\makeatletter
\renewcommand*{\@biblabel}[1]{\hfill#1.}
\renewcommand*\l@section{\@dottedtocline{1}{1em}{1em}}
%%% Здесь может начать бухтеть и давать предупреждения, пока не найдет в тексте хотя бы один рисунок и/или хотя бы одну таблицу. Если это бесит, то закомментируй. 
\renewcommand{\thefigure}{\thesection.\arabic{figure}} % Формат рисунка секция.номер
\renewcommand{\thetable}{\thesection.\arabic{table}}   % Формат таблицы секция.номер
\def\redeflsection{\def\l@section{\@dottedtocline{1}{0em}{10em}}}
\makeatother

\renewcommand{\baselinestretch}{1.4} % Полуторный межстрочный интервал
\parindent 1.27cm % Абзацный отступ

\sloppy             % Избавляемся от переполнений
\hyphenpenalty=1000 % Частота переносов
\clubpenalty=10000  % Запрещаем разрыв страницы после первой строки абзаца
\widowpenalty=10000 % Запрещаем разрыв страницы после последней строки абзаца

% Отступы у страниц
\usepackage[left=3cm,right=1.5cm,bottom=2cm,top=2cm]{geometry}

% Списки
\usepackage{enumitem}
\setlist[enumerate,itemize]{leftmargin=12.7mm} % Отступы в списках

\makeatletter
    \AddEnumerateCounter{\asbuk}{\@asbuk}{м)}
\makeatother
\setlist{nolistsep}                           % Нет отступов между пунктами списка
\renewcommand{\labelitemi}{--}                % Маркер списка --
\renewcommand{\labelenumi}{\asbuk{enumi})}    % Список второго уровня
\renewcommand{\labelenumii}{\arabic{enumii})} % Список третьего уровня

% Содержание
\usepackage{tocloft}
\renewcommand{\cfttoctitlefont}{\hspace{0.38\textwidth}\MakeTextUppercase} % СОДЕРЖАНИЕ
\renewcommand{\cftsecfont}{\hspace{0pt}}            % Имена секций в содержании не жирным шрифтом
\renewcommand\cftsecleader{\cftdotfill{\cftdotsep}} % Точки для секций в содержании
\renewcommand\cftsecpagefont{\mdseries}             % Номера страниц не жирные
\setcounter{tocdepth}{3}                            % Глубина оглавления, до subsubsection

% Список иллюстративного материала
\renewcommand{\cftloftitlefont}{\hspace{0.17\textwidth}\MakeTextUppercase}
\renewcommand{\cftfigfont}{Рисунок }
\addto\captionsrussian{\renewcommand\listfigurename{Список иллюстративного материала}}

% Список табличного материала
\renewcommand{\cftlottitlefont}{\hspace{0.2\textwidth}\MakeTextUppercase}
\renewcommand{\cfttabfont}{Таблица }
\addto\captionsrussian{\renewcommand\listtablename{Список табличного материала}}

% Нумерация страниц посередине снизу
\usepackage{fancyhdr}
\pagestyle{fancy}
\fancyhf{}
\cfoot{\textrm{\thepage}}
\fancyheadoffset{0mm}
\fancyfootoffset{0mm}
\setlength{\headheight}{17pt}
\renewcommand{\headrulewidth}{0pt}
\renewcommand{\footrulewidth}{0pt}
\fancypagestyle{plain}{
    \fancyhf{}
    \cfoot{\textrm{\thepage}}
}

% Формат подрисуночных надписей
\RequirePackage{caption}
\DeclareCaptionLabelSeparator{defffis}{ -- } % Разделитель
\captionsetup[figure]{justification=centering, labelsep=defffis, format=plain} % Подпись рисунка по центру

\captionsetup[table]{justification=raggedright, labelsep=defffis, format=plain, singlelinecheck=false} % Подпись таблицы слева
\addto\captionsrussian{\renewcommand{\figurename}{Рисунок}} % Имя фигуры

% Пользовательские функции
\newcommand{\addimg}[4]{ % Добавление одного рисунка
    \begin{figure}
        \centering
        \includegraphics[width=#2\linewidth]{#1}
        \caption{#3} \label{#4}
    \end{figure}
}
\newcommand{\addimghere}[4]{ % Добавить рисунок непосредственно в это место
    \begin{figure}[H]
        \centering
        \includegraphics[width=#2\linewidth]{#1}
        \caption{#3} \label{#4}
    \end{figure}
}
\newcommand{\addtwoimghere}[6]{ % Вставка двух рисунков
    \begin{figure}[H]
        \centering
        \includegraphics[width=#3\linewidth]{#1}
        \hfill
        \includegraphics[width=#4\linewidth]{#2}
        \caption{#5} \label{#6}
    \end{figure}
}

\usepackage{wasysym} % Пришлось добавить для символов из статьи
\usepackage{placeins} % Для FloatBarrier

% Заголовки секций в оглавлении в верхнем регистре
\usepackage{textcase}
\makeatletter
\let\oldcontentsline\contentsline
\def\contentsline#1#2{
    \expandafter\ifx\csname l@#1\endcsname\l@section
        \expandafter\@firstoftwo
    \else
        \expandafter\@secondoftwo
    \fi
    {\oldcontentsline{#1}{\MakeTextUppercase{#2}}}
    {\oldcontentsline{#1}{#2}}
}
\makeatother

% Оформление заголовков
\usepackage[compact,explicit]{titlesec}
\titleformat{\section}{}{}{12.5mm}{\centering{\thesection\quad\MakeTextUppercase{#1}}\vspace{1.5em}}
\titleformat{\subsection}[block]{\vspace{1em}}{}{12.5mm}{\thesubsection\quad#1\vspace{1em}}
\titleformat{\subsubsection}[block]{\vspace{1em}\normalsize}{}{12.5mm}{\thesubsubsection\quad#1\vspace{1em}}
\titleformat{\paragraph}[block]{\normalsize}{}{12.5mm}{\MakeTextUppercase{#1}}

% Секции без номеров (введение, заключение...), вместо section*{}
\newcommand{\anonsection}[1]{
    \phantomsection % Корректный переход по ссылкам в содержании
    \paragraph{\centerline{{#1}}\vspace{1em}}
    \addcontentsline{toc}{section}{#1}
}

% Секция для аннотации (она не включается в содержание)
\newcommand{\annotation}[1]{
    \paragraph{\centerline{{#1}}\vspace{1em}}
}

% Секция для списка иллюстративного материала
\newcommand{\lof}{
    \phantomsection
    \listoffigures
    \addcontentsline{toc}{section}{\listfigurename}
}

% Секция для списка табличного материала
\newcommand{\lot}{
    \phantomsection
    \listoftables
    \addcontentsline{toc}{section}{\listtablename}
}

% Секции для приложений
\newcommand{\appsection}[1]{
    \phantomsection
    \paragraph{\centerline{{#1}}}
    \addcontentsline{toc}{section}{{#1}}
}

% Библиография: отступы и межстрочный интервал
\makeatletter
\renewenvironment{thebibliography}[1]
    {\section*{\refname}
        \list{\@biblabel{\@arabic\c@enumiv}}
           {\settowidth\labelwidth{\@biblabel{#1}}
            \leftmargin\labelsep
            \itemindent 16.7mm
            \@openbib@code
            \usecounter{enumiv}
            \let\p@enumiv\@empty
            \renewcommand\theenumiv{\@arabic\c@enumiv}
        }
        \setlength{\itemsep}{0pt}
    }
\makeatother

\usepackage{lastpage} % Подсчет количества страниц
\setcounter{page}{1}  % Начало нумерации страниц

\begin{document}
    %%% При загрузке текста ВКР на сайт МФТИ титульная страница
    %%% будет сгенерирована автоматически, 
    %%% скачиваешь её, добавляешь в папку inc. 
    %%% Когда будет готово, включишь эту строчку. 
    %%% \includepdf[]{inc/0-titlepage.pdf} % Титульная страница
    
    %%% В наш год положение о ВКР требовала именно такой 
    %%% последовательности секций работы.
    \annotation{Аннотация}

Тема выпускной квалификационной работы бакалавра --- <<Сравнительный анализ монолитного и микросервисного подходов к разработке web приложений в условиях низкой загрузки>>.

Здесь текст аннотации: $\dots$. //TODO

Выпускная квалификационная работа бакалавра изложена на \pageref{LastPage} листах, включает N таблиц, M рисунков, L приложений, K литературных источников. //TODO

Ключевые слова: $\dots$. //TODO

% Обязательно добавляем это в конце каждой секции, чтобы 
% обеспечить переход на новую страницу
\clearpage % Аннотация
    \tableofcontents % Содержание 
    \clearpage
    
    \anonsection{Основные понятия}

    \anonsubsection{Монолитная архитектура}
        \textbf{Монолитная архитектура} — это традиционная модель программного обеспечения, которая представляет собой единый модуль, работающий автономно и независимо от других приложений. Такое ПО обладает единой базой кода, в которой объединены все бизнес-задачи в виде одного большого унифицированного блока кода (монолита).
    
    \anonsubsection{Микросервисная архитектура}
        \textbf{Микросервисная архитектура} - модель программного обеспечения, при которой используются небольшие модульные сервисы. Основная концепция архитектуры в том, чтобы разделить сложное приложение на несколько небольших автономных и управляемых компонентов. Каждый микросервис имеет собственный набор кода, базу данных и API для взаимодействия с другими сервисами.

    \anonsubsection{Стенды}
        \textbf{Стенды} - собирательное название сред разработки, позволяющих изолировать некоторый этап жизни проекта. Так, основные виды включают тестовый стенд - среда разработки, где разработчики могут проводить любые действия, не влияя на реальную систему, и стенд прода - непосредственно запущенное приложение, с которым взаимодействуют пользователи.

    \anonsubsection{User Story}
        \textbf{User Story} - это способ записи требований к ПО, который помогает команде разработки формализовать требования клиента, не прибегая к излишнему формализму. Представляет собой подробное описание сценариев поведения, которые ожидаются при взаимодействии пользователя с ПО.
 
    \anonsubsection{Docker}
        \textbf{Docker} -  программное обеспечение, предназначенное для автоматизации развёртывания и управления приложениями. Достигается это через создание "контейнеров" для приложения, в которых хранятся их окружения и зависимости и который может быть развёрнут на любой Linux-системе.
    
    \anonsubsection{GitHub}
        \textbf{GitHub} - крупнейший сервис для хранения исходного кода программного обеспечения, основанный на системе контроля версий Git. Помимо контроля версий и хранения кода позволяет автоматизировать тестирование и деплой с помощью CI/CD.
    
    \anonsubsection{Java}
        \textbf{Java} -  строго типизированный объектно-ориентированный язык программирования; приложения Java обычно транслируются в специальный байт-код, в результате чего достигается кросскомпиляция. Один из самых популярных языков программирования для разработки Web-приложений  (2-е место в рейтингах IEEE Spectrum (2020) и TIOBE (2021)).
    
    \anonsubsection{PSQL (PostgreSQL)}
        \textbf{PSQL (PostgreSQL)} - самая большая свободная объектно-реляционная система управления базами данных, активно поддерживаемая по сегодняшний день.
    
    \anonsubsection{Vue.js}
        \textbf{Vue.js} - JavaScript-фреймворк с открытым исходным кодом для создания пользовательских интерфейсов.
    
    \anonsubsection{Keycloak}
        \textbf{Keycloak} - продукт с открытым кодом, предназначенный для реализации регистрации и авторизации с возможностью добавления ролевой модели. Keycloak используется для написания минимального количества кода, при этом обеспечивая безопасность при аутентификации.
    

% Обязательно добавляем это в конце каждой секции, чтобы 
% обеспечить переход на новую страницу
\clearpage % Перечень сокращений и условных обозначений
    \anonsection{Введение}
    //TODO

% Обязательно добавляем это в конце каждой секции, чтобы 
% обеспечить переход на новую страницу
\clearpage % Введение
    %%% Литературный обзор %%%
\section{Литературный обзор}

\subsection{Первый раздел}
    Здесь, допустим, мы хотим сразу же начать с цитирования \cite{newton2014newton}. 
    
    \subsubsection{Первый подраздел}
        А ещё здесь хотим такое цитирование \cite{granqvist1976ultrafine, нагаев1992малые}. 

\subsection{Второй раздел}
    Добавим несколько картинок (Рисунок \ref{fig:01-example-1}). 
    
    \addtwoimghere{01-example-1.png}{01-example-2.png}{0.49}{0.49}{В картинках также работают ссылки. Пусть \cite{newton2014newton}.}{fig:01-example-1}
    
    На приложение в тексте обязательно должна быть сделана ссылка ---  \hyperlink{app-a}{Приложение А}.
    
        
% Обязательно добавляем это в конце каждой секции, чтобы 
% обеспечить переход на новую страницу
\clearpage % Литературный обзор
    %%% Материалы и методы %%%
\section{Материалы и методы}

\subsection{Первый раздел}
    Сюда добавим какую-нибудь таблицу (Таблица \ref{table:01-coeffs}). 
    
    \begin{table}[H]
        \caption{Таблица коэффициентов} \label{table:01-coeffs}
        \begin{tabular}{|p{0.6cm}|p{4.9cm}|p{4.5cm}|p{4cm}|}
        \hline \# & Колонка 1 & Колонка 2 & Колонка 3 \\
        \hline 1 & Один & $f(x) + c$ & $4.1 $ \\
        \hline 2 & Два & $f(x) - a$ & $4.2 $ \\
        \hline 3 & Три & $f(x) \ \sim \ b$ & $4.3$ \\
        \hline
        \end{tabular}
    \end{table}
    
    \subsubsection{Первый подраздел}
        Первый подраздел первый подраздел первый подраздел первый подраздел первый подраздел первый подраздел первый подраздел первый подраздел первый подраздел первый подраздел первый подраздел первый подраздел первый подраздел 

\subsection{Второй раздел}
    Второй раздел второй раздел второй раздел второй раздел второй раздел второй раздел второй раздел второй раздел второй раздел второй раздел второй раздел второй раздел второй раздел второй раздел второй раздел второй раздел второй раздел второй раздел 
    
        
% Обязательно добавляем это в конце каждой секции, чтобы 
% обеспечить переход на новую страницу
\clearpage % Материалы и методы
    %%% Результаты %%%
\section{Результаты}

\subsection{Первый раздел}
    Первый раздел первый раздел первый раздел первый раздел первый раздел первый раздел первый раздел первый раздел первый раздел первый раздел первый раздел первый раздел первый раздел первый раздел первый раздел первый раздел первый раздел первый раздел первый раздел первый раздел первый раздел первый раздел первый раздел первый раздел. 
    
    \subsubsection{Первый подраздел}
        Первый подраздел первый подраздел первый подраздел первый подраздел первый подраздел первый подраздел первый подраздел первый подраздел первый подраздел первый подраздел первый подраздел первый подраздел первый подраздел 

\subsection{Второй раздел}
    Второй раздел второй раздел второй раздел второй раздел второй раздел второй раздел второй раздел второй раздел второй раздел второй раздел второй раздел второй раздел второй раздел второй раздел второй раздел второй раздел второй раздел второй раздел 
    
        
% Обязательно добавляем это в конце каждой секции, чтобы 
% обеспечить переход на новую страницу
\clearpage % Результаты эксперимента 
    %%% Выводы %%%
\section{Выводы}

\subsection{Первый раздел}
    Первый раздел первый раздел первый раздел первый раздел первый раздел первый раздел первый раздел первый раздел первый раздел первый раздел первый раздел первый раздел первый раздел первый раздел первый раздел первый раздел первый раздел первый раздел первый раздел первый раздел первый раздел первый раздел первый раздел первый раздел. 
    
    \subsubsection{Первый подраздел}
        Первый подраздел первый подраздел первый подраздел первый подраздел первый подраздел первый подраздел первый подраздел первый подраздел первый подраздел первый подраздел первый подраздел первый подраздел первый подраздел 

\subsection{Второй раздел}
    Второй раздел второй раздел второй раздел второй раздел второй раздел второй раздел второй раздел второй раздел второй раздел второй раздел второй раздел второй раздел второй раздел второй раздел второй раздел второй раздел второй раздел второй раздел 
    
        
% Обязательно добавляем это в конце каждой секции, чтобы 
% обеспечить переход на новую страницу
\clearpage % Выводы
    \anonsection{Заключение}

    Цель данного исследования - сравнительный анализ монолитного и микросервисного подходов к разработке веб-приложений в условиях низкой загрузки. Исследование было направлено на выявление преимуществ и недостатков каждой архитектуры на примере сравнения двух проектов, чтобы предоставить рекомендации по выбору подходящей архитектуры для малых и средних проектов.

    В ходе работы было представлено комплексное описание монолитной и микросервисной архитектур, включающее в себя несколько пунктов. Во-первых, были проанализированы уже существующие исследования с целью получения понимания о текущих наработках по данной теме и подтверждения новизны данной работы. Во-вторых, были подробно описаны два проекта, разработанные командой программистов под моим руководством, имеющие разную архитектуру. Наконец, были рассмотрены ключевые вопросы, связанные с различными аспектами разработки и эксплуатации веб-приложений и проведён сравнительный анализ.

    Для сравнительного анализа были выбраны критерии так, чтобы максимально отразить условия низкой загрузки, но при этом дать всестороннюю оценку эффективности и применимости каждой архитектуры. Ниже приведём краткую сводку по результатам сравнения в рамках каждого критерия.

    \begin{itemize}
        \item Команда и скорость разработки: Монолитная архитектура часто способствует более быстрой разработке, так как вся команда работает над одной кодовой базой. Это упрощает коммуникацию и координацию, что может быть критически важным для малых команд.
        \item Совместимость: Несмотря на то, что каждый микросервис можно адаптировать для взаимодействия с различными системами и сервисами, монолитная архитектура обеспечивает прямое, непосредственное взаимодействие, что важнее для упрощения разработки.
        \item Переносимость: Микросервисы обеспечивают лучшую переносимость, так как каждый сервис можно развернуть и масштабировать независимо на разных платформах и в различных средах, используя контейнеризацию и оркестрацию. Тем не менее контейнеризация применима так же и к монолитной архитектуре, что может уменьшить риски и повысить удобство управления приложением.
        \item Наблюдаемость: Монолитная архитектура позволяет естественным образом аггрегировать логи, а также производить отладку с помощью одного специализированного приложения в едином месте, что упрощает разработку. Тогда как микросервисная архитектура требует дополнительных усилий для настройки и управления мониторингом.
        \item Потребление ресурсов: В условиях низкой загрузки монолитная архитектура часто оказывается более эффективной, так как отсутствуют накладные расходы на межсервисные коммуникации. Однако микросервисы могут быть лучше оптимизированы в долгосрочной перспективе благодаря возможности независимого масштабирования.
    \end{itemize}

    \medskip
    
    Для небольших и средних проектов с низкой загрузкой важно внимательно оценить текущие и будущие потребности. Монолитная архитектура может быть предпочтительной, когда важны скорость разработки и простота управления, а микросервисная архитектура подходит, если планируется масштабирование проекта и интеграция с другими системами в будущем.
    
    Также монолитная архитектура оптимальна для проектов с ограниченными ресурсами и маленькими командами разработчиков, где приоритетом являются быстрая разработка и простота управления. Это хороший выбор для проектов, не предусматривающих значительный рост или масштабирование в ближайшее время, таких как стартапы и прототипы, поскольку позволяет быстро вывести продукт на рынок. В отличие от этого, микросервисная архитектура рекомендуется для проектов, требующих высокой гибкости и возможности масштабирования отдельных компонентов; также она подходит для систем, нуждающихся в частых обновлениях и интеграции с внешними сервисами. Проекты с распределенной разработкой, где участвуют несколько команд, тоже могут выиграть от использования микросервисного подхода.
    
    При выборе архитектуры необходимо учитывать не только текущие требования, но и долгосрочные цели проекта. Если предполагается быстрый рост и развитие, микросервисы могут быть более подходящими, в то время как для проектов с четко определенными и стабильными требованиями монолитная архитектура может быть эффективнее. Важно также учитывать навыки команды и наличие ресурсов для поддержки выбранной архитектуры, поскольку микросервисная архитектура требует большей подготовки и лучшего понимания принципов проектирования.
    
    Будущие исследования могут быть направлены на изучение производительности и масштабируемости монолитных и микросервисных архитектур в условиях высокой загрузки. Также стоит исследовать влияние новых технологий и инструментов на выбор архитектуры. Наконец, важно рассмотреть гибридные подходы, объединяющие преимущества обеих архитектур, разработав третий проект и проведя аналогичный анализ.
    
    %Для малых и средних проектов с низкой загрузкой рекомендуется тщательно оценить текущие и будущие потребности проекта. Монолитная архитектура может быть предпочтительнее в случаях, когда важны скорость разработки и простота управления, в то время как микросервисная архитектура может подойти, если проект планируется масштабировать и интегрировать с другими системами в будущем.
    
    %Монолитная архитектура оптимальна для проектов с ограниченными ресурсами и небольшими командами разработчиков, где важны скорость разработки и простота управления. Также она подходит для проектов, которые не предполагают значительного роста или масштабирования в ближайшем будущем, например, для стартапов и прототипов, так как она позволит разработчикам быстро вывести продукт на рынок. Микросервисная архитектура, наоборот, рекомендуется для проектов, где требуется высокая гибкость и возможность масштабирования отдельных компонентов, а также она подходит для систем, которые требуют частых обновлений и интеграции с внешними сервисами. Проекты, в которых ведётся распределенная разработка с участием нескольких команд, также могут выиграть от использования микросервисного подхода.
    
    %Также при выборе архитектуры важно учитывать не только текущие требования проекта, но и его долгосрочные цели. Если проект предполагает быстрый рост и развитие, то микросервисы могут быть более подходящими, а для проектов с четко определенными и стабильными требованиями монолитная архитектура может оказаться более эффективной. Важно также учитывать навыки команды и наличие ресурсов для поддержки выбранной архитектуры, поскольку микросервисная архитектура требует большей подготовки и лучшего понимания принципов создания архитектуры.
    
    %Будущие исследования могут быть направлены на изучение монолитных и микросервисных архитектур в условиях высокой загрузки, чтобы оценить их производительность и масштабируемость в более требовательных сценариях. Также целесообразно исследовать влияние новых технологий и инструментов на выбор архитектуры. Наконец, важно рассмотреть гибридные подходы, которые могут объединять преимущества обеих архитектур, для чего можно разработать третий проект и провести аналогичный анализ.

% Обязательно добавляем это в конце каждой секции, чтобы 
% обеспечить переход на новую страницу
\clearpage % Заключение
    
    \renewcommand{\section}[2]{\anonsection{Библиографический список}}
%%% Подключение списка литературы по госту 2008 года
\bibliographystyle{ugost2008l}
% Подключаем сам файл с литературой (разрешение .bib)

%%% В ФАЙЛЕ ДЛЯ РУССКИХ ИСТОЧНИКОВ ДОБАВЛЯЕМ language={russian}
\bibliography{references}

\label{sec:bibliography}

% Обязательно добавляем это в конце каждой секции, чтобы 
% обеспечить переход на новую страницу
\clearpage % Подключение библиографии
    
    \appsection{Приложение А} \hypertarget{app-a}{\label{app-a}}

//TODO

\clearpage % Подключение приложения

\end{document}
