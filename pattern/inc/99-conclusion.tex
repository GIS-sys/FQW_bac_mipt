\anonsection{Заключение}

    Цель данного исследования - сравнительный анализ монолитного и микросервисного подходов к разработке веб-приложений в условиях низкой загрузки. Исследование было направлено на выявление преимуществ и недостатков каждой архитектуры на примере сравнения двух проектов, чтобы предоставить рекомендации по выбору подходящей архитектуры для малых и средних проектов.

    В ходе работы было представлено комплексное описание монолитной и микросервисной архитектур, включающее в себя несколько пунктов. Во-первых, были проанализированы уже существующие исследования с целью получения понимания о текущих наработках по данной теме и подтверждения новизны данной работы. Во-вторых, были подробно описаны два проекта, разработанные командой программистов под моим руководством, имеющие разную архитектуру. Наконец, были рассмотрены ключевые вопросы, связанные с различными аспектами разработки и эксплуатации веб-приложений и проведён сравнительный анализ.

    Для сравнительного анализа были выбраны критерии так, чтобы максимально отразить условия низкой загрузки, но при этом дать всестороннюю оценку эффективности и применимости каждой архитектуры. Ниже приведём краткую сводку по результатам сравнения в рамках каждого критерия.

    \begin{itemize}
        \item Команда и скорость разработки: Монолитная архитектура часто способствует более быстрой разработке, так как вся команда работает над одной кодовой базой. Это упрощает коммуникацию и координацию, что может быть критически важным для малых команд.
        \item Совместимость: Несмотря на то, что каждый микросервис можно адаптировать для взаимодействия с различными системами и сервисами, монолитная архитектура обеспечивает прямое, непосредственное взаимодействие, что важнее для упрощения разработки.
        \item Переносимость: Микросервисы обеспечивают лучшую переносимость, так как каждый сервис можно развернуть и масштабировать независимо на разных платформах и в различных средах, используя контейнеризацию и оркестрацию. Тем не менее контейнеризация применима так же и к монолитной архитектуре, что может уменьшить риски и повысить удобство управления приложением.
        \item Наблюдаемость: Монолитная архитектура позволяет естественным образом аггрегировать логи, а также производить отладку с помощью одного специализированного приложения в едином месте, что упрощает разработку. Тогда как микросервисная архитектура требует дополнительных усилий для настройки и управления мониторингом.
        \item Потребление ресурсов: В условиях низкой загрузки монолитная архитектура часто оказывается более эффективной, так как отсутствуют накладные расходы на межсервисные коммуникации. Однако микросервисы могут быть лучше оптимизированы в долгосрочной перспективе благодаря возможности независимого масштабирования.
    \end{itemize}

    \medskip
    
    Для небольших и средних проектов с низкой загрузкой важно внимательно оценить текущие и будущие потребности. Монолитная архитектура может быть предпочтительной, когда важны скорость разработки и простота управления, а микросервисная архитектура подходит, если планируется масштабирование проекта и интеграция с другими системами в будущем.
    
    Также монолитная архитектура оптимальна для проектов с ограниченными ресурсами и маленькими командами разработчиков, где приоритетом являются быстрая разработка и простота управления. Это хороший выбор для проектов, не предусматривающих значительный рост или масштабирование в ближайшее время, таких как стартапы и прототипы, поскольку позволяет быстро вывести продукт на рынок. В отличие от этого, микросервисная архитектура рекомендуется для проектов, требующих высокой гибкости и возможности масштабирования отдельных компонентов; также она подходит для систем, нуждающихся в частых обновлениях и интеграции с внешними сервисами. Проекты с распределенной разработкой, где участвуют несколько команд, тоже могут выиграть от использования микросервисного подхода.
    
    При выборе архитектуры необходимо учитывать не только текущие требования, но и долгосрочные цели проекта. Если предполагается быстрый рост и развитие, микросервисы могут быть более подходящими, в то время как для проектов с четко определенными и стабильными требованиями монолитная архитектура может быть эффективнее. Важно также учитывать навыки команды и наличие ресурсов для поддержки выбранной архитектуры, поскольку микросервисная архитектура требует большей подготовки и лучшего понимания принципов проектирования.
    
    Будущие исследования могут быть направлены на изучение производительности и масштабируемости монолитных и микросервисных архитектур в условиях высокой загрузки. Также стоит исследовать влияние новых технологий и инструментов на выбор архитектуры. Наконец, важно рассмотреть гибридные подходы, объединяющие преимущества обеих архитектур, разработав третий проект и проведя аналогичный анализ.
    
    %Для малых и средних проектов с низкой загрузкой рекомендуется тщательно оценить текущие и будущие потребности проекта. Монолитная архитектура может быть предпочтительнее в случаях, когда важны скорость разработки и простота управления, в то время как микросервисная архитектура может подойти, если проект планируется масштабировать и интегрировать с другими системами в будущем.
    
    %Монолитная архитектура оптимальна для проектов с ограниченными ресурсами и небольшими командами разработчиков, где важны скорость разработки и простота управления. Также она подходит для проектов, которые не предполагают значительного роста или масштабирования в ближайшем будущем, например, для стартапов и прототипов, так как она позволит разработчикам быстро вывести продукт на рынок. Микросервисная архитектура, наоборот, рекомендуется для проектов, где требуется высокая гибкость и возможность масштабирования отдельных компонентов, а также она подходит для систем, которые требуют частых обновлений и интеграции с внешними сервисами. Проекты, в которых ведётся распределенная разработка с участием нескольких команд, также могут выиграть от использования микросервисного подхода.
    
    %Также при выборе архитектуры важно учитывать не только текущие требования проекта, но и его долгосрочные цели. Если проект предполагает быстрый рост и развитие, то микросервисы могут быть более подходящими, а для проектов с четко определенными и стабильными требованиями монолитная архитектура может оказаться более эффективной. Важно также учитывать навыки команды и наличие ресурсов для поддержки выбранной архитектуры, поскольку микросервисная архитектура требует большей подготовки и лучшего понимания принципов создания архитектуры.
    
    %Будущие исследования могут быть направлены на изучение монолитных и микросервисных архитектур в условиях высокой загрузки, чтобы оценить их производительность и масштабируемость в более требовательных сценариях. Также целесообразно исследовать влияние новых технологий и инструментов на выбор архитектуры. Наконец, важно рассмотреть гибридные подходы, которые могут объединять преимущества обеих архитектур, для чего можно разработать третий проект и провести аналогичный анализ.

% Обязательно добавляем это в конце каждой секции, чтобы 
% обеспечить переход на новую страницу
\clearpage