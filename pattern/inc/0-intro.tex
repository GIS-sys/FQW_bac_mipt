\anonsection{Введение}

Разработка веб-приложений представляет собой процесс создания программных продуктов, которые функционируют на веб-платформе и доступны через интернет-браузеры. История веб-разработки началась в 1990-х годах с появлением первых веб-страниц, которые представляли собой простые статические документы. С течением времени веб-технологии эволюционировали, добавляя интерактивные элементы, базы данных и сложные пользовательские интерфейсы, что привело к возникновению современных динамических веб-приложений.

Важным аспектом разработки веб-приложений является выбор архитектуры, которая определяет структуру и взаимодействие различных компонентов системы. Этот выбор существенно влияет на производительность, масштабируемость, надежность и управляемость приложения. В течение первых 15 лет монолитная архитектура, где все компоненты приложения тесно связаны и работают как единое целое, была практически единственным подходом, применяемым в веб разработке любого масштаба и уровня. Тем не менее, при увеличении количества пользователей, а также при повышении доступных мощностей серверов и уменьшении задержки при передачи информации через интернет, монолитный подход стал приводить к громоздким программным системам, взаимодействие между которыми зачастую полагалось на стандартизованные тяжеловесные протоколы.

В результате в 2010х годах приобрела популярность микросервисная архитектура, которая, напротив, разбивает приложение на множество независимых сервисов, каждый из которых выполняет определенную функцию и может разрабатываться и развертываться отдельно. Такой подход позволил достичь большей гибкости и масштабируемости, заложив прочный фундамент для современных IT-гигантов вроде Google и Amazon.

Но процесс выбора архитектуры для веб-приложения зависит от множества факторов, включая размер и сложность проекта, прогнозируемую нагрузку, требования к масштабируемости и ресурсы команды разработчиков. В настоящее время как в мире, так и в России, наблюдается широкое распространение обеих архитектур. Монолитные архитектуры по-прежнему популярны для небольших и средних проектов, где важны скорость разработки и простота управления. Микросервисные архитектуры набирают популярность в крупных проектах и компаниях, где требуется высокая гибкость и масштабируемость.

Цель данной работы — провести сравнительный анализ монолитного и микросервисного подходов к разработке веб-приложений в условиях низкой загрузки. Для достижения этой цели работа будет разбита на три основных раздела. В первом разделе будет подробнее рассказано про микросервисную и монолитную архитектуры,  конкретные причины, почему выбор архитектуры до сих пор важен и актуален, а также какие исследования уже проведены и почему фокус данного исследования был сделан на низкой загруженности. Во втором разделе будут подробно описаны два проекта, один с монолитной, второй с микросервисной архитектурами. Наконец, в третьем разделе будут приведены несколько критериев, по которым можно проводить сравнивать проекты, и будет проведён сам сравнительный анализ для выявления достоинств и недостатков каждого подхода.

Выводы данной работы будут полезны для разработчиков и менеджеров проектов, стоящих перед выбором архитектуры для своих веб-приложений. Результаты исследования помогут определить, какой из подходов более эффективен в условиях низкой загрузки и какие критерии следует учитывать при принятии решения. Это, в свою очередь, позволит улучшить качество и эффективность разработки веб-приложений в малых и средних проектах.

% Обязательно добавляем это в конце каждой секции, чтобы 
% обеспечить переход на новую страницу
\clearpage