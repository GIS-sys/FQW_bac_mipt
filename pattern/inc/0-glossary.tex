\anonsection{Основные понятия}

    \anonsubsection{Монолитная архитектура}
        \textbf{Монолитная архитектура} — это традиционная модель программного обеспечения, которая представляет собой единый модуль, работающий автономно и независимо от других приложений. Такое ПО обладает единой базой кода, в которой объединены все бизнес-задачи в виде одного большого унифицированного блока кода (монолита).
    
    \anonsubsection{Микросервисная архитектура}
        \textbf{Микросервисная архитектура} - модель программного обеспечения, при которой используются небольшие модульные сервисы. Основная концепция архитектуры в том, чтобы разделить сложное приложение на несколько небольших автономных и управляемых компонентов. Каждый микросервис имеет собственный набор кода, базу данных и API для взаимодействия с другими сервисами.

    \anonsubsection{Стенды}
        \textbf{Стенды} - собирательное название сред разработки, позволяющих изолировать некоторый этап жизни проекта. Так, основные виды включают тестовый стенд - среда разработки, где разработчики могут проводить любые действия, не влияя на реальную систему, и стенд прода - непосредственно запущенное приложение, с которым взаимодействуют пользователи.

    \anonsubsection{User Story}
        \textbf{User Story} - это способ записи требований к ПО, который помогает команде разработки формализовать требования клиента, не прибегая к излишнему формализму. Представляет собой подробное описание сценариев поведения, которые ожидаются при взаимодействии пользователя с ПО.
 
    \anonsubsection{Docker}
        \textbf{Docker} -  программное обеспечение, предназначенное для автоматизации развёртывания и управления приложениями. Достигается это через создание "контейнеров" для приложения, в которых хранятся их окружения и зависимости и который может быть развёрнут на любой Linux-системе.
    
    \anonsubsection{GitHub}
        \textbf{GitHub} - крупнейший сервис для хранения исходного кода программного обеспечения, основанный на системе контроля версий Git. Помимо контроля версий и хранения кода позволяет автоматизировать тестирование и деплой с помощью CI/CD.
    
    \anonsubsection{Java}
        \textbf{Java} -  строго типизированный объектно-ориентированный язык программирования; приложения Java обычно транслируются в специальный байт-код, в результате чего достигается кросскомпиляция. Один из самых популярных языков программирования для разработки Web-приложений  (2-е место в рейтингах IEEE Spectrum (2020) и TIOBE (2021)).
    
    \anonsubsection{PSQL (PostgreSQL)}
        \textbf{PSQL (PostgreSQL)} - самая большая свободная объектно-реляционная система управления базами данных, активно поддерживаемая по сегодняшний день.
    
    \anonsubsection{Vue.js}
        \textbf{Vue.js} - JavaScript-фреймворк с открытым исходным кодом для создания пользовательских интерфейсов.
    
    \anonsubsection{Keycloak}
        \textbf{Keycloak} - продукт с открытым кодом, предназначенный для реализации регистрации и авторизации с возможностью добавления ролевой модели. Keycloak используется для написания минимального количества кода, при этом обеспечивая безопасность при аутентификации.
    

% Обязательно добавляем это в конце каждой секции, чтобы 
% обеспечить переход на новую страницу
\clearpage